\documentclass[a4j,12]{jarticle}
\usepackage{exp1}

%\pagestyle{empty}
%%%%%%%%%%%%%%%%%%%%%
%%%レポート作成者情報の入力
%%%班
\def\group{グループ番号}
%%%学籍番号
\def\idnumber{1315193T}
%%%氏名
\def\name{薛 宇航}
%%%実験月日月
\def\expmontho{月}
\def\expdayo{日}
%%%実験月日
\def\expmontht{月}
\def\expdayt{日}
%%%実験月日
\def\expmonthth{月}
\def\expdayth{日}
%%%レポート提出月日
\def\deadlinem{01}
\def\deadlined{13}
%%%共同実験者1
\def\groupmembero{1305194T 中山 峻一}
%%%共同実験者2 居なかった場合には{}だけにする.
\def\groupmembert{1385195T 藤原 巧}
%%%共同実験者3 居た場合には{}内に学籍番号と氏名を入力する.
\def\groupmemberh{1365196T 和田 佳大}
%%%%%%%%%%%%%%%%%%%


\begin{document}

\title{
\vspace{1cm}{\LARGE 平成27年度}\\
% \vspace{1em}
\begin{spacing}{1.5}
{\Huge ディジタル信号処理}\\
{\huge --ニューラルネットワーク--}
\end{spacing}
\author{}
\date{}
}%%表紙の情報、、入力忘れないように
\maketitle
\vspace{5cm}
{\Large
\begin{spacing}{1.2}
%%%\noindent\hspace{20em}\underline{班:\group 班}\\
%%%\hspace{20em}\underline{実験日時:\expmontho 月\expdayo 日}\\
%%%\hspace{20em}\underline{    :\expmontht 月\expdayt 日}\\
%%%\hspace{20em}\underline{    :\expmonthth 月\expdayth 日}\\
\hspace{20em}\underline{レポート提出日:\deadlinem 月\deadlined 日}\\
\hspace{20em}\underline{学籍番号:\idnumber}\\
\hspace{20em}\underline{氏  名:\name}\\
\end{spacing}
}
\thispagestyle{empty}
\newpage
\setcounter{page}{1}      %ページ番号リセット


\section{課題内容}
ネット上に置いてあるclass1.datとclass2.datの二つの学習データにそれぞれ100個のデータが含んでる。class1とclass2のd0たをプロットすると、class1の一次元目は1の周辺の値、二次元目が2の周辺の値であることがわかる。\par
また、class2の一次元目は2の周辺の値、二次元目は1の周辺の値であることがわかる。ここで、三層のニューラルネットワークを考える。このニューラルネットワークを用いて、2クラス識別が可能な識別きを構築する。重みの初期値がrand()を使い、後処理で平均0となるようにして、横軸は繰り返し回数で、d[n]とy[n]の二乗誤差をプロットし、収束している様子を確認する。\par
500回繰り返し学習後、ニューラルネットワークの出力値$y_1[n]$、$y_2[n]$を示す。テストデータに対してニューラルネットワークの出力値をプロットして、各サンプルがどのクラスに属しているかを調べよ。
\section{レポート内容}
\subsection{レポート結果}
資料により、重みの更新式は以下の式(1)と式(2)のように一般化した:
\begin{eqnarray} 
  n\_w=^2_{kj}=w^2_{kj}+2(d_k[n]-g_k[n])*f'(u_k)*g_j\\
  n\_w=^1_{ji}=w^1_{jk}+{\sum_k(d_k-g_k)*f'(u_k)*w_{kj}}*f'(u_j)*x_i[n]
\end{eqnarray}%%数式
ここで、特にf(x)はシグモイド関数で、f'(x)はこの関数の微分で、それぞれ以下の式(3)、(4)のようになる。
\begin{eqnarray}
  f(x)=\frac{1}{1+e^{-x}}\\
  f'(x)=f(x)*(1-f(x))
\end{eqnarray}

以上より、500回繰り返し学習後、d[n]とy[n]の二乗誤差erをgnuplotで出力した結果は以下の図1のようになる。
\begin{figure}[htpb]
  \begin{center}
    \includegraphics[width=15cm]{picture/error.eps}
    \caption{学習回数と誤差の関係}
    \label{micon}
  \end{center}
\end{figure}
そして、学習データに対するニューラルネットワークの出力値のプロット結果は以下の図2のようになる。
\begin{figure}[htpb]
  \begin{center}
    \includegraphics[width=15cm]{picture/output_y.eps}
    \caption{学習データに対するニューラルネットワークの出力値}
    \label{micon}
  \end{center}
\end{figure}
そして、この重みを用いて、テストデータの出力は以下の図3のようになる。
\begin{figure}[htpb]
  \begin{center}
    \includegraphics[width=15cm]{picture/output_test.eps}
    \caption{テストデータに対するニューラルネットワークの出力値}
    \label{micon}
  \end{center}
\end{figure}
図3からわかるように、大体50個目のデータを境目にして、前半はクラス1に属し、後半20個データはクラス2に属す。


\subsection{使用するプログラム}
\begin{lstlisting}[caption=プログラム,label=ラベル]
#include <stdio.h>
#include <math.h>
#include <stdlib.h>
#include <time.h>

float function(float u){
  return 1/(1+exp(-u));
}

float diff_function(float u){
  return (1-function(u))*function(u);
}

int main()
{
  char class1_name[128]="class1.dat";
  char class2_name[128]="class2.dat";
  char test_file[128]="test.dat";
  char result_y1[128]="result_1.dat";
  char result_y2[128]="result_2.dat";
  char error_file[128]="error.dat";
  char out_node1[128]="y1.dat";
  char out_node2[128]="y2.dat";
  float cls[200];
  float clss[140];
  float class1_x[2][100],class2_x[2][100];
  float class1_y[2][100],class2_y[2][100];
  float test_x[2][200],test_y[2][200];

  float class1_d[2],class2_d[2];
  float w1[2][2],w2[2][2],n_w1[2][2],n_w2[2][2],sum;

  float class1_uj[2],class1_uk[2],class1_gj[2];
  float class2_uj[2],class2_uk[2],class2_gj[2];
  float test_uj[2],test_uk[2],test_gj[2];
  float err[500];
  
  int num_update;
  
  int n,i,j,k;
  FILE *file_input;
  FILE *file_output;

  //class1を読み込んで、それをclass1_xに代入する。
  if((file_input=fopen(class1_name,"rb"))==NULL){
    fprintf(stderr,"cannot read %s. \n",class1_name);exit(-1);
  }
  fread(cls,sizeof(float),200,file_input);
  fclose(file_input);
  for(i=0;i<100;i++){
    for(j=0;j<2;j++){
      class1_x[j][i]=cls[i*2+j];
    }
  }
  //class2を読み込んで、それをclass2_xに代入する。
  if((file_input=fopen(class2_name,"rb"))==NULL){
    fprintf(stderr,"cannot read %s. \n",class2_name);exit(-1);
  }
  fread(cls,sizeof(float),200,file_input);
  fclose(file_input);
  for(i=0;i<100;i++){
    for(j=0;j<2;j++){
      class2_x[j][i]=cls[i*2+j];
    }
  }

  //testを読み込んで、それをtest_xに代入する。
  if((file_input=fopen(test_file,"rb"))==NULL){
    fprintf(stderr,"cannot read %s. \n",test_file);exit(-1);
  }
  n=fread(clss,sizeof(float),140,file_input);
  fclose(file_input);
  for(i=0;i<70;i++){
    for(j=0;j<2;j++){
      test_x[j][i]=clss[i*2+j];
    }
  }


  
  //実行する度に違う乱数を得られる。信憑性を高まる。
  srand((unsigned)time(NULL));
  //w1の初期化
  sum=0.0;
  for(i=0;i<2;i++){
    for(j=0;j<2;j++){
      w1[i][j]=0.1*(double)rand()/RAND_MAX;
      sum+=w1[i][j];
    }
  }
  sum=sum/4;//乱数の平均を求める。
  for(i=0;i<2;i++){
    for(j=0;j<2;j++){
      w1[i][j]=w1[i][j]-sum;//平均を0にするために
    }
  }
  //w2の初期化
  sum=0.0;
  for(i=0;i<2;i++){
    for(j=0;j<2;j++){
      w2[i][j]=0.1*(double)rand()/RAND_MAX;
      sum+=w2[i][j];
    }
  }
  sum=sum/4;//乱数の平均を求める。
  for(i=0;i<2;i++){
    for(j=0;j<2;j++){
      w2[i][j]=w2[i][j]-sum;//平均を0にするために      
    }
  }
  //教師信号の初期化
  class1_d[1-1]=1;//class1のd[1]
  class1_d[2-1]=0;//class1のd[2]
  class2_d[1-1]=0;//class2のd[1]
  class2_d[2-1]=1;//class2のd[2]

  for(num_update=0;num_update<500;num_update++){
    for(i=0;i<2;i++){
      for(j=0;j<2;j++){
	n_w1[i][j]=w1[i][j];
	n_w2[i][j]=w2[i][j];
      }
    }
    err[num_update]=0.0;
    for(n=0;n<100;n++){
      //変数の初期化
      for(i=0;i<2;i++){
	class1_uj[i]=0;	class2_uj[i]=0;
	class1_uk[i]=0; class2_uk[i]=0;
	class1_gj[i]=0; class2_gj[i]=0;
		
     }
      //重みが変わる時に変数の変化
      for(j=0;j<2;j++){
	for(i=0;i<2;i++){
	  class1_uj[j]+=w1[j][i]*class1_x[i][n];
	  class2_uj[j]+=w1[j][i]*class2_x[i][n];
	}
	class1_gj[j]=function(class1_uj[j]);
	class2_gj[j]=function(class2_uj[j]);
      }
      for(k=0;k<2;k++){
	for(j=0;j<2;j++){
	  class1_uk[k]+=w2[k][j]*class1_gj[j];
	  class2_uk[k]+=w2[k][j]*class2_gj[j];
	}
	class1_y[k][n]=function(class1_uk[k]);
	class2_y[k][n]=function(class2_uk[k]);
      }
      err[num_update]+=(class1_y[0][n]-class1_d[0])*(class1_y[0][n]-class1_d[0])+(class1_y[1][n]-class1_d[1])*(class1_y[1][n]-class1_d[1]);
      err[num_update]+=(class2_y[0][n]-class2_d[0])*(class2_y[0][n]-class2_d[0])+(class2_y[1][n]-class2_d[1])*(class2_y[1][n]-class2_d[1]);

      //2-3層の間の重みw2を更新
      for(k=0;k<2;k++){
	for(j=0;j<2;j++){


	  n_w2[k][j]+=0.005*(class1_d[k]-class1_y[k][n])*diff_function(class1_uk[k])*class1_gj[j];
	  n_w2[k][j]+=0.005*(class2_d[k]-class2_y[k][n])*diff_function(class2_uk[k])*class2_gj[j];
	}
      }
      //1-2層の間の重みw1を更新
      for(j=0;j<2;j++){
	for(i=0;i<2;i++){
	  sum=0;
	  for(k=0;k<2;k++){
	    sum+=(class1_d[k]-class1_y[k][n])*diff_function(class1_uk[k])*w2[k][j];
	  }
	  n_w1[j][i]+=0.005*sum*diff_function(class1_uj[j])*class1_x[i][n];
	  
	  
	  sum=0;
	  for(k=0;k<2;k++){
	    sum+=(class2_d[k]-class2_y[k][n])*diff_function(class2_uk[k])*w2[k][j];
	  }
	  n_w1[j][i]+=0.005*sum*diff_function(class2_uj[j])*class2_x[i][n];

	}
      }
      
    }
    //更新した重みn_wをwに代入する
    for(i=0;i<2;i++){
      for(j=0;j<2;j++){
	w1[i][j]=n_w1[i][j];
	w2[i][j]=n_w2[i][j];
	
      }
    }
  }
  //誤差を保存する
  if((file_output = fopen(error_file,"w")) == NULL){
    fprintf(stderr, "Cannot write %s\n", error_file);  exit(-1);
  }
  for(n=0;n<500;n++){
    fprintf(file_output,"%lf \n",err[n]); 
  }
  fclose(file_output);
  //ノードのy1値をoutputする
  if((file_output = fopen(out_node1,"w")) == NULL){
    fprintf(stderr, "Cannot write %s\n", out_node1);  exit(-1);
  }
  for(n=0;n<200;n++){
    if(n<100){
      fprintf(file_output,"%lf \n",class1_y[0][n]); 
    }
    else{
      fprintf(file_output,"%lf \n",class2_y[0][n-100]); 
    }
  }
  fclose(file_output);
  //ノードのy2値をoutputする
  if((file_output = fopen(out_node2,"w")) == NULL){
    fprintf(stderr, "Cannot write %s\n", out_node2);  exit(-1);
  }
  for(n=0;n<200;n++){
    if(n<100){
      fprintf(file_output,"%lf \n",class1_y[1][n]); 
    }
    else{
      fprintf(file_output,"%lf \n",class2_y[1][n-100]); 
    }
  }
  fclose(file_output);


  //テストデータの出力を計算
  for(n=0;n<70;n++){
    for(i=0;i<2;i++){
	test_uj[i]=0;
	test_uk[i]=0;
	test_gj[i]=0;	
     }

    //重みが変わる時に変数の変化
      for(j=0;j<2;j++){
	for(i=0;i<2;i++){
	  test_uj[j]+=w1[j][i]*test_x[i][n];
	}
	test_gj[j]=function(test_uj[j]);
      }
      for(k=0;k<2;k++){
	for(j=0;j<2;j++){
	  test_uk[k]+=w2[k][j]*test_gj[j];
	}
	test_y[k][n]=function(test_uk[k]);
      }
  }
  //テスト出力のy1値をoutputする
  if((file_output = fopen(result_y1,"w")) == NULL){
    fprintf(stderr, "Cannot write %s\n", result_y1);  exit(-1);
  }
  for(n=0;n<70;n++){
      fprintf(file_output,"%lf \n",test_y[0][n]); 
  }
  fclose(file_output);

  //テスト出力のy2値をoutputする
  if((file_output = fopen(result_y2,"w")) == NULL){
    fprintf(stderr, "Cannot write %s\n", result_y2);  exit(-1);
  }
  for(n=0;n<70;n++){
      fprintf(file_output,"%lf \n",test_y[1][n]); 
  }
  fclose(file_output);

  
  return 0;
}  
    


\end{lstlisting}
%参考文献
\begin{thebibliography}{9}
 \bibitem {jikken} http://www.me.cs.scitec.kobe-u.ac.jp/~takigu/class/2015/work5/nn.pdf
\end{thebibliography}
\end{document}
